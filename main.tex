% LaTeX Paper Template (English Version)
% This file is automatically generated from main_J.tex
\documentclass[12pt, a4paper]{article}

% English support
% \usepackage[utf8]{inputenc} % UTF-8エンコーディング(LuaLaTeX/XeLaTeXでは不要)
\usepackage[english]{babel}  % English support

% Basic packages
\usepackage{amsmath, amssymb}  % Mathematics
\usepackage{graphicx}          % Figures
\usepackage{hyperref}          % Hyperlinks
\usepackage{cite}              % Citation optimization

% Page settings
\usepackage[top=25mm, bottom=25mm, left=25mm, right=25mm]{geometry}

% Title information
\title{Research on Metabolism and SCD}
\author{Satoshi Yamaguchi}
\date{\today}

\begin{document}

\maketitle

\begin{abstract}
Write your abstract here in English. Briefly summarize the background, objectives, methods, main results, and conclusions of your research.

Usually 150-250 words.
\end{abstract}

\section{Introduction}
\subsection{Background}
Write the research background here in English.

\subsection{Objectives}
Write the research objectives here in English.

To cite references, use the cite command like \cite{example_article}.

To cite multiple references, write like \cite{example_article, example_book}.

\section{Methods}
\subsection{Study Design}
Write the study design here in English.

\subsection{Subjects}
Write about the research subjects here in English.

\subsection{Analysis Methods}
Write the analysis methods here in English.

% Mathematicsの例
% \begin{equation}
%   E = mc^2
% \end{equation}

\section{Results}
\subsection{Baseline Characteristics}
Write the baseline characteristics results here in English.

\subsection{Main Results}
Write the main results here in English.

% Figure 1: Metabolic rate comparison
\begin{figure}[htbp]
  \centering
  \includegraphics[width=0.7\textwidth]{figures/figure1_metabolism_comparison.pdf}
  \caption{Comparison of metabolic rates between Group A and Group B. Error bars represent standard deviation.}
  \label{fig:metabolism}
\end{figure}

As shown in Figure \ref{fig:metabolism}, Group B exhibited significantly higher metabolic rates compared to Group A.

% Figure 2: Correlation plot
\begin{figure}[htbp]
  \centering
  \includegraphics[width=0.7\textwidth]{figures/figure2_correlation.pdf}
  \caption{Correlation between age and SCD risk score. A significant positive correlation was observed (R² = 0.67, p < 0.001).}
  \label{fig:correlation}
\end{figure}

A strong positive correlation was found between age and SCD risk score (Figure \ref{fig:correlation}).

% Figure 3: Survival curve
\begin{figure}[htbp]
  \centering
  \includegraphics[width=0.7\textwidth]{figures/figure3_survival_curve.pdf}
  \caption{Kaplan-Meier survival curves. The treatment group showed significantly better prognosis compared to the control group (Log-rank p < 0.05).}
  \label{fig:survival}
\end{figure}

As shown in Figure \ref{fig:survival}, the treatment group demonstrated significantly better survival rates compared to the control group.

% Example of table
% \begin{table}[htbp]
%   \centering
%   \caption{Table caption}
%   \label{tab:table1}
%   \begin{tabular}{lcc}
%     \hline
%     Item & Group A & Group B \\
%     \hline
%     Value 1 & 10.2 & 12.5 \\
%     Value 2 & 8.7 & 9.3 \\
%     \hline
%   \end{tabular}
% \end{table}

\section{Discussion}
\subsection{Main Findings}
Write the main findings here in English.

\subsection{Comparison with Previous Studies}
Write the comparison with previous studies here in English.

\subsection{Limitations of This Study}
Write the limitations of this study here in English.

\section{Conclusion}
Write your conclusion briefly in English. Summarize the main findings of the research and their clinical and academic significance.

% References
\bibliographystyle{plain}  % Referencesスタイル(plain, unsrt, alpha, abbrv など)
\bibliography{references}   % references.bibファイルを参照

\end{document}

