% ============================================================
% LaTeX Paper Template (English Version) - SAMPLE
% ============================================================
% This is a TEMPLATE file with sample content
% Replace the sample text with your actual research content
%
% This file is automatically generated from main_J.tex using:
%   ./translate.sh
%
% To start writing:
%   1. Edit main_J.tex (Japanese version)
%   2. Run ./translate.sh to update this file
%   3. Compile with ./compile.sh
% ============================================================

\documentclass[12pt, a4paper]{article}

% English support
\usepackage[english]{babel}  % English support

% Basic packages
\usepackage{amsmath, amssymb}  % Mathematics
\usepackage{graphicx}          % Figures
\usepackage{hyperref}          % Hyperlinks
\usepackage{cite}              % Citation optimization

% Page settings
\usepackage[top=25mm, bottom=25mm, left=25mm, right=25mm]{geometry}

% ============================================================
% Title information - REPLACE WITH YOUR INFORMATION
% ============================================================
\title{Research on Metabolism and SCD}
\author{Satoshi Yamaguchi}
\date{\today}

\begin{document}

\maketitle

\begin{abstract}
Write your abstract here in English. Briefly summarize the background, objectives, methods, main results, and conclusions of your research.

Usually 150-250 words.
\end{abstract}

\section{Introduction}
\subsection{Background}
Write the research background here in English.

\subsection{Objectives}
Write the research objectives here in English.

To cite references, use the cite command like \cite{example_article}.

To cite multiple references, write like \cite{example_article, example_book}.

\section{Methods}
\subsection{Study Design}
Write the study design here in English.

\subsection{Subjects}
Write about the research subjects here in English.

\subsection{Analysis Methods}
Write the analysis methods here in English.

% Mathematicsの例
% \begin{equation}
%   E = mc^2
% \end{equation}

\section{Results}
\subsection{Baseline Characteristics}
Write the baseline characteristics results here in English.

\subsection{Main Results}
Write the main results here in English.

% 図1: 代謝率の比較
\begin{figure}[htbp]
  \centering
  \includegraphics[width=0.7\textwidth]{figures/figure1_metabolism_comparison.pdf}
  \caption{代謝率の比較。Group AとGroup Bの代謝率を示す。エラーバーは標準偏差を表す。}
  \label{fig:metabolism}
\end{figure}

図\ref{fig:metabolism}に示すように、Group BはGroup Aと比較して有意に高い代謝率を示した。

% 図2: 相関図
\begin{figure}[htbp]
  \centering
  \includegraphics[width=0.7\textwidth]{figures/figure2_correlation.pdf}
  \caption{年齢とSCDリスクスコアの相関。有意な正の相関が認められた(R² = 0.67, p < 0.001)。}
  \label{fig:correlation}
\end{figure}

年齢とSCDリスクスコアの間には強い正の相関が認められた(図\ref{fig:correlation})。

% 図3: 生存曲線
\begin{figure}[htbp]
  \centering
  \includegraphics[width=0.7\textwidth]{figures/figure3_survival_curve.pdf}
  \caption{Kaplan-Meier生存曲線。治療群は対照群と比較して有意に良好な予後を示した(Log-rank p < 0.05)。}
  \label{fig:survival}
\end{figure}

図\ref{fig:survival}に示すように、治療群は対照群と比較して有意に良好な生存率を示した。



% Table 1: Basic table
\begin{table}[htbp]
  \centering
  \caption{Patient Baseline Characteristics}
  \label{tab:baseline}
  \begin{tabular}{lcc}
    \hline
    Variable & Control (n=50) & Treatment (n=50) \\
    \hline
    Age (years) & 65.2 $\pm$ 8.3 & 64.8 $\pm$ 7.9 \\
    Male (\%) & 60.0 & 58.0 \\
    BMI (kg/m²) & 24.5 $\pm$ 3.2 & 24.1 $\pm$ 3.0 \\
    Systolic BP (mmHg) & 135.2 $\pm$ 15.6 & 133.8 $\pm$ 14.2 \\
    \hline
  \end{tabular}
\end{table}

Table \ref{tab:baseline} shows the baseline characteristics of patients.

% Table 2: More detailed table with grouped headers
\begin{table}[htbp]
  \centering
  \caption{Comparison of Treatment Effects (Mean $\pm$ SD)}
  \label{tab:outcomes}
  \begin{tabular}{lccc}
    \hline
    & Baseline & 3 months & p value \\
    \hline
    \multicolumn{4}{l}{\textbf{Control Group}} \\
    Metabolic rate (kcal/day) & 1650 $\pm$ 220 & 1620 $\pm$ 210 & 0.24 \\
    Body weight (kg) & 68.5 $\pm$ 12.3 & 68.8 $\pm$ 12.5 & 0.56 \\
    \hline
    \multicolumn{4}{l}{\textbf{Treatment Group}} \\
    Metabolic rate (kcal/day) & 1680 $\pm$ 230 & 1850 $\pm$ 240 & \textbf{0.001} \\
    Body weight (kg) & 69.2 $\pm$ 11.8 & 67.5 $\pm$ 11.2 & \textbf{0.03} \\
    \hline
  \end{tabular}
\end{table}

Table \ref{tab:outcomes} shows detailed treatment effects. Significant p values are shown in bold.

% Table 3: Multivariate analysis results
\begin{table}[htbp]
  \centering
  \caption{Multivariate Analysis of SCD Risk Factors}
  \label{tab:multivariate}
  \begin{tabular}{lccc}
    \hline
    Risk Factor & Hazard Ratio & 95\% CI & p value \\
    \hline
    Age (per 10 years) & 1.45 & 1.22--1.73 & \textbf{$<$0.001} \\
    Male sex & 1.32 & 0.98--1.78 & 0.07 \\
    BMI (per 5 kg/m²) & 1.18 & 0.95--1.47 & 0.14 \\
    Hypertension & 1.67 & 1.23--2.27 & \textbf{0.001} \\
    Diabetes & 1.89 & 1.34--2.66 & \textbf{$<$0.001} \\
    Smoking & 1.52 & 1.15--2.01 & \textbf{0.003} \\
    \hline
  \end{tabular}
\end{table}

As shown in Table \ref{tab:multivariate}, age, hypertension, diabetes, and smoking were identified as independent SCD risk factors.

\section{Discussion}
\subsection{Main Findings}
Write the main findings here in English.

\subsection{Comparison with Previous Studies}
Write the comparison with previous studies here in English.

\subsection{Limitations of This Study}
Write the limitations of this study here in English.

\section{Conclusion}
Write your conclusion briefly in English. Summarize the main findings of the research and their clinical and academic significance.

% References
\bibliographystyle{plain}  % Referencesスタイル(plain, unsrt, alpha, abbrv など)
\bibliography{references}   % references.bibファイルを参照

\end{document}

