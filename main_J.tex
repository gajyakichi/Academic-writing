% ============================================================
% LaTeX論文テンプレート(日本語版)- サンプル
% ============================================================
% これはサンプル内容を含むテンプレートファイルです
% サンプルテキストを実際の研究内容に置き換えてください
%
% このファイルを編集後、以下のコマンドで英語版を生成できます:
%   ./translate.sh
%
% 執筆の流れ:
%   1. このファイル(main_J.tex)を日本語で編集
%   2. ./compile_J.sh で日本語版PDFを生成
%   3. ./translate.sh で英語版(main.tex)を生成
%   4. ./compile.sh で英語版PDFを生成
% ============================================================

\documentclass[12pt, a4paper]{article}

% 日本語対応(upLaTeX、pLaTeXの場合)
\usepackage[japanese]{babel}  % 日本語サポート

% 基本パッケージ
\usepackage{amsmath, amssymb}  % 数式
\usepackage{graphicx}          % 図の挿入
\usepackage{hyperref}          % ハイパーリンク
\usepackage{cite}              % 引用の最適化

% ページ設定
\usepackage[top=25mm, bottom=25mm, left=25mm, right=25mm]{geometry}

% ============================================================
% タイトル情報
% ============================================================
\title{メタボリックシンドロームおよび前メタボリックシンドロームと心臓突然死の地域相関:全国特定健診データを用いた研究}
\author{
  Satoshi Yamaguchi\textsuperscript{1,2}, Michio Shimabukuro\textsuperscript{1}\\
  \small \textsuperscript{1}Department of Diabetes, Endocrinology and Metabolism, School of Medicine, Fukushima Medical University\\
  \small \textsuperscript{2}Department of Cardiology, Northern Okinawa Anshin Clinic
}
\date{\today}

\begin{document}

\maketitle

\begin{abstract}
メタボリックシンドローム(MetS)は、心血管疾患(CVD)や死亡率のリスク増大との関連から、世界的に公衆衛生上の重大な問題として浮上している。本研究では、日本の全都道府県を対象に、特定健診データを活用し、MetSおよび前MetS(Pre-MetS)の有病率と心臓突然死(SCD)の発生率との相関を検討した。特に、致死的なショック適応リズム(shockable rhythm)と非ショック適応リズムそれぞれについて分析を行い、地域差がリスクに及ぼす影響を明らかにすることを目的とした。その結果、MetSおよびPre-MetSの有病率は、shockable rhythmによるSCDの発生率と有意な正の相関を示したが、non-shockable rhythmとの相関は認められなかった。これらの知見は、地域ごとのMetS有病率に基づくリスク評価が、SCDの予防戦略に有用である可能性を示唆している。
\end{abstract}

\section{Introduction}\label{introduction}

メタボリックシンドローム(MetS)は、心血管疾患(CVD)や死亡率のリスク増大との関連から、世界的に公衆衛生上の重大な問題として浮上している。メタボリックシンドロームは、中心性肥満、高血圧、脂質異常症、インスリン抵抗性を含む一連の病態として定義され、世界中の医療システムに多面的な課題となっている\cite{alberti2009harmonizing,grundy2005diagnosis}。長寿と高齢化として知られている日本でさえも、MetSの有病率はここ数十年、生活様式や食習慣の変化と並行して着実に増加している\cite{haruyama2020incidence,hiratsuka2021higher}。

日本においては、特定健診がこれらのリスク因子の早期発見と予防に貢献している。MetSの有病率には地域差があるが、有病率の高さが直接的に心臓突然死(SCD)の発生率に関連しているかは不明である。

また、近年では、MetSの前段階である「前メタボリック症候群(Pre-Metabolic Syndrome, Pre-MetS)」のリスクにも注目が集まっている。前メタボリック症候群とは、MetS診断基準に達していないものの、MetSの要素が存在する状態を指す。この段階であっても、高血圧や耐糖能異常が生じていることが多く、これらが心筋のリモデリングや自律神経系の不均衡を引き起こし、心室性不整脈などの致死性不整脈のリスクを高める可能性が指摘されている。このことから、前メタボリック症候群の段階での介入も心血管リスクの低減に重要であると考えられる。

本研究では、日本の全都道府県を対象に、特定健診データを活用し、MetSおよび前MetSの有病率とSCDの発生率との相関を検討する。特に、致死的なショック適応リズム(shockable rhythm:心室細動や心室頻拍によるもの)と非ショック適応リズム(非心室性不整脈など)それぞれについて分析を行い、地域差がリスクに及ぼす影響を明らかにすることを目的としている。

\section{Methods}\label{methods}

\subsection{Enrollment}\label{enrollment}

JCS-ReSS研究は院外心停止の全日本登録であるAll-Japan Utstein Registry of the Fire and Disaster Management Agencyデータを用いた観察研究である。本研究の対象期間は2021年4月から2022年3月までの1年間で発生した院外心停止患者43,565人をリクルートした。心原性群(Cardiogenic group, n = 26,281)と非心原性群(Non-cardiogenic group, n = 17,284)に分けた。さらに心原性群の26,281人中で初期波形がVFとpulseless VTであったshockable rhythm群(n = 4,119)とPEAやAsystoleを含むそれ以外の波形のnon-shockable rhythm群(n = 22,162)に分類した。

\subsection{Data Collection}\label{data-collection}

心停止データはAll-Japan Utstein Registryという前向き、人口ベース、全国登録データベースを使用した。

特定健康診査(特定健診)は、日本の国民皆保険制度における保険者が提供するヘルスチェックプログラムで、年に1回、40歳から74歳を対象にしたメタボリックシンドロームのスクリーニングである。人口統計は政府統計データを利用した\footnote{\url{https://www.e-stat.go.jp/stat-search/files?page=1&layout=datalist&toukei=00200524&tstat=000000090001&cycle=7&year=20210&month=0&tclass1=000001011679}}。

データは厚生労働省のウェブサイトより取得した\footnote{\url{https://www.data.go.jp/data/dataset/mhlw_20210614_0065}}。

\subsection{Definition of Metabolic Syndrome and Pre-Metabolic Syndrome}\label{definition-of-metabolic-syndrome-and-pre-metabolic-syndrome}

メタボリック症候群(MetS)は以下の定義とした。男性は腹囲85cm以上、女性は90cm以上の腹囲肥満と、高血糖、高血圧、脂質異常症のいずれか2つを満たした場合をメタボリック症候群とした。腹囲肥満および、高血糖、高血圧、脂質異常症の1つを満たした場合を前メタボリック症候群(Pre-MetS)と定義した。高血糖は空腹時血糖$\geq$110mg/dLまたはHbA1c$\geq$5.5\%、高血圧は収縮期血圧$\geq$130mmHgまたは拡張期血圧$\geq$85mmHg、脂質異常症は中性脂肪$\geq$150mg/dLまたはHDL-C$<$40mg/dLと定義した。

\subsection{The Percentage of Pre- and Metabolic Syndrome}\label{the-percentage-of-pre--and-metabolic-syndrome}

特定健診データには特定健康診査の受診者数とMetS数とPre-MetS数を含んでおり、それぞれの都道府県のMetSとPre-MetSの割合を算出した。

\subsection{The Definition of Shockable Rhythm and Non-Shockable Rhythm}\label{the-definition-of-shockable-rhythm-and-non-shockable-rhythm}

院外心停止ウツタインデータベースは日本全国の消防隊が収集した院外心停止のデータベースである。心停止発生時に消防隊の隊員が患者情報および発生状況を記載する。心原性、非心原性を院外心停止の状況から消防隊が判断する。また、自動体外除細動器および心電図モニターを用い、心電図の初期リズムを判断している。

\subsection{The Estimation of Incidence of OHCA}\label{the-estimation-of-incidence-of-ohca}

院外心停止の推定発生率については、院外心停止を特定健診対象者推定値で除して10万人あたりの発生率を計算した。

\subsection{Statistical Analysis}\label{statistical-analysis}

各都道府県のMetSおよびPre-MetSの有病率と心停止cases/10\textsuperscript{6} populationの相関係数をSpearman's $\rho$を用いて評価した。

\section{Results}\label{results}

メタボリックシンドロームおよび前メタボリックシンドロームの特定健診受診者内の有病率を示した。MetSの有病率は最低の静岡県の13.97\%から最高の沖縄県の19.19\%であった。Pre-MetSの有病率は最低の新潟県の25.19\%から最高の沖縄県の34.09\%であった。

\subsection{Correlation Between Prevalence of Metabolic Syndrome and Out-of-Hospital Cardiac Arrest Cases}\label{correlation-between-prevalence-of-metabolic-syndrome-and-out-of-hospital-cardiac-arrest-cases}

MetSとPre-MetSの有病率とshockable rhythmによる院外心停止の発生率は有意な相関関係であった(r=0.389, P=0.007)。MetSとPre-MetSの有病率とnon-shockableおよびnon-cardiogenicの院外心停止は有意な相関関係は認められなかった。

MetS、Pre-MetS、それぞれの有病率とshockable rhythmによる院外心停止の発生率は有意な相関関係であった(MetS, r=0.343, P=0.018; Pre-MetS, r=0.347, P=0.017)。MetS、Pre-MetS、それぞれの有病率とnon-shockableおよびnon-cardiogenicの院外心停止は有意な相関関係はなかった。

\section{Discussion}\label{discussion}

本研究では以下の2つの主要な知見を得た。(1)MetSとPre-MetSはそれぞれ、shockable rhythmの発生と正の相関関係を認めた。(2)MetSとPre-MetSはそれぞれ、non-shockable rhythmと有意な相関関係は認めなかった。

(1)に関して考察を深める。MetSは心血管イベントのリスクを高める主要な要因として注目されてきた\cite{dekker2005metabolic}。またメタボリック症候群に含まれる高血圧、脂質異常、高血糖などの要素が、動脈硬化性疾患だけでなく、心室性不整脈のリスク因子となることが報告されている\cite{tirandi2022role}。

MetSが直接的にVFやVTを誘発し、SCDに至るリスクを高める可能性が示唆されている。例えば、慢性炎症や酸化ストレス、自律神経の異常を引き起こし、致死的不整脈を促進するメカニズムが提案されている\cite{paoletti2006metabolic,masenga2023mechanisms}。

(2)に関して考える。Non-shockable rhythm(無脈性電気活動や心静止)のリスクファクターは、shockable rhythm(心室細動や心室頻拍)とは異なり、主に全身性の生理機能低下や酸素不足に関連する要因が中心となる。具体的なリスクファクターとしては、低酸素血症、重度の低血圧、電解質異常、代謝異常、心タンポナーデなどが挙げられる。急性の電気的異常によるリズムとは異なるメカニズムが働くため、metabolic syndromeと関連しなかったと考えられる。

\subsection{Limitations}\label{limitations}

心停止の初期リズムを消防隊が院外で判断している。実際に本研究で観察されたshockable rhythmの患者がmetabolic syndromeを有していたかは調べられていない。

\section{Conclusion}\label{conclusion}

本研究により、メタボリックシンドローム(MetS)および前メタボリックシンドローム(Pre-MetS)が、shockable rhythm(心室細動や心室頻拍)による心臓突然死(SCD)のリスクを高める可能性が示唆された。本研究の結果は、地域ごとのMetS有病率に基づくリスク評価が、SCDの予防戦略に有用である可能性を示しており、今後の特定健診データを活用したさらなる研究の基盤となることが期待される。

% 参考文献
\bibliographystyle{plain}  % 参考文献スタイル(plain, unsrt, alpha, abbrv など)
\bibliography{references}   % references.bibファイルを参照

\end{document}
