% LaTeX論文テンプレート(日本語版)
% このファイルを編集すると、自動的に main.tex に英訳されます
\documentclass[12pt, a4paper]{article}

% 日本語対応(upLaTeX、pLaTeXの場合)
% \usepackage[utf8]{inputenc} % UTF-8エンコーディング(LuaLaTeX/XeLaTeXでは不要)
\usepackage[japanese]{babel}  % 日本語サポート

% 基本パッケージ
\usepackage{amsmath, amssymb}  % 数式
\usepackage{graphicx}          % 図の挿入
\usepackage{hyperref}          % ハイパーリンク
\usepackage{cite}              % 引用の最適化

% ページ設定
\usepackage[top=25mm, bottom=25mm, left=25mm, right=25mm]{geometry}

% タイトル情報
\title{代謝とSCDに関する研究}
\author{山口怜}
\date{\today}

\begin{document}

\maketitle

\begin{abstract}
ここに日本語で要旨を記述します。研究の背景、目的、方法、主要な結果、結論を簡潔にまとめます。

通常、150〜250語程度で記述します。
\end{abstract}

\section{序論}
\subsection{研究背景}
ここに日本語で研究の背景を記述します。

\subsection{研究目的}
ここに日本語で研究の目的を記述します。

参考文献を引用する場合は、\cite{example_article}のようにciteコマンドを使用します。

複数の文献を引用する場合は、\cite{example_article, example_book}のように記述できます。

\section{方法}
\subsection{研究デザイン}
ここに日本語で研究デザインを記述します。

\subsection{対象}
ここに日本語で研究対象を記述します。

\subsection{解析方法}
ここに日本語で解析方法を記述します。

% 数式の例
% \begin{equation}
%   E = mc^2
% \end{equation}

\section{結果}
\subsection{基本特性}
ここに日本語で基本特性の結果を記述します。

\subsection{主要な結果}
ここに日本語で主要な結果を記述します。

% 図の挿入例
% \begin{figure}[htbp]
%   \centering
%   \includegraphics[width=0.8\textwidth]{figure1.pdf}
%   \caption{図のキャプション}
%   \label{fig:figure1}
% \end{figure}

% 表の例
% \begin{table}[htbp]
%   \centering
%   \caption{表のキャプション}
%   \label{tab:table1}
%   \begin{tabular}{lcc}
%     \hline
%     項目 & 群A & 群B \\
%     \hline
%     値1 & 10.2 & 12.5 \\
%     値2 & 8.7 & 9.3 \\
%     \hline
%   \end{tabular}
% \end{table}

\section{考察}
\subsection{主要な知見}
ここに日本語で主要な知見を記述します。

\subsection{先行研究との比較}
ここに日本語で先行研究との比較を記述します。

\subsection{本研究の限界}
ここに日本語で本研究の限界を記述します。

\section{結論}
ここに日本語で結論を簡潔に記述します。研究の主要な発見と、その臨床的・学術的意義をまとめます。

% 参考文献
\bibliographystyle{plain}  % 参考文献スタイル(plain, unsrt, alpha, abbrv など)
\bibliography{references}   % references.bibファイルを参照

\end{document}
