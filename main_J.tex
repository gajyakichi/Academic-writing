% LaTeX論文テンプレート(日本語版)
% このファイルを編集すると、自動的に main.tex に英訳されます
\documentclass[12pt, a4paper]{article}

% 日本語対応(upLaTeX、pLaTeXの場合)
% \usepackage[utf8]{inputenc} % UTF-8エンコーディング(LuaLaTeX/XeLaTeXでは不要)
\usepackage[japanese]{babel}  % 日本語サポート

% 基本パッケージ
\usepackage{amsmath, amssymb}  % 数式
\usepackage{graphicx}          % 図の挿入
\usepackage{hyperref}          % ハイパーリンク
\usepackage{cite}              % 引用の最適化

% ページ設定
\usepackage[top=25mm, bottom=25mm, left=25mm, right=25mm]{geometry}

% タイトル情報
\title{代謝とSCDに関する研究}
\author{山口怜}
\date{\today}

\begin{document}

\maketitle

\begin{abstract}
ここに日本語で要旨を記述します。研究の背景、目的、方法、主要な結果、結論を簡潔にまとめます。

通常、150〜250語程度で記述します。
\end{abstract}

\section{序論}
\subsection{研究背景}
ここに日本語で研究の背景を記述します。

\subsection{研究目的}
ここに日本語で研究の目的を記述します。

参考文献を引用する場合は、\cite{example_article}のようにciteコマンドを使用します。

複数の文献を引用する場合は、\cite{example_article, example_book}のように記述できます。

\section{方法}
\subsection{研究デザイン}
ここに日本語で研究デザインを記述します。

\subsection{対象}
ここに日本語で研究対象を記述します。

\subsection{解析方法}
ここに日本語で解析方法を記述します。

% 数式の例
% \begin{equation}
%   E = mc^2
% \end{equation}

\section{結果}
\subsection{基本特性}
ここに日本語で基本特性の結果を記述します。

\subsection{主要な結果}
ここに日本語で主要な結果を記述します。

% 図1: 代謝率の比較
\begin{figure}[htbp]
  \centering
  \includegraphics[width=0.7\textwidth]{figures/figure1_metabolism_comparison.pdf}
  \caption{代謝率の比較。Group AとGroup Bの代謝率を示す。エラーバーは標準偏差を表す。}
  \label{fig:metabolism}
\end{figure}

図\ref{fig:metabolism}に示すように、Group BはGroup Aと比較して有意に高い代謝率を示した。

% 図2: 相関図
\begin{figure}[htbp]
  \centering
  \includegraphics[width=0.7\textwidth]{figures/figure2_correlation.pdf}
  \caption{年齢とSCDリスクスコアの相関。有意な正の相関が認められた(R² = 0.67, p < 0.001)。}
  \label{fig:correlation}
\end{figure}

年齢とSCDリスクスコアの間には強い正の相関が認められた(図\ref{fig:correlation})。

% 図3: 生存曲線
\begin{figure}[htbp]
  \centering
  \includegraphics[width=0.7\textwidth]{figures/figure3_survival_curve.pdf}
  \caption{Kaplan-Meier生存曲線。治療群は対照群と比較して有意に良好な予後を示した(Log-rank p < 0.05)。}
  \label{fig:survival}
\end{figure}

図\ref{fig:survival}に示すように、治療群は対照群と比較して有意に良好な生存率を示した。


% 表1: 基本的な表
\begin{table}[htbp]
  \centering
  \caption{患者の基本特性}
  \label{tab:baseline}
  \begin{tabular}{lcc}
    \hline
    項目 & 対照群 (n=50) & 治療群 (n=50) \\
    \hline
    年齢(歳) & 65.2 $\pm$ 8.3 & 64.8 $\pm$ 7.9 \\
    男性(\%) & 60.0 & 58.0 \\
    BMI(kg/m²) & 24.5 $\pm$ 3.2 & 24.1 $\pm$ 3.0 \\
    収縮期血圧(mmHg) & 135.2 $\pm$ 15.6 & 133.8 $\pm$ 14.2 \\
    \hline
  \end{tabular}
\end{table}

表\ref{tab:baseline}に患者の基本特性を示す。

% 表2: より詳細な表(複数列のヘッダー付き)
\begin{table}[htbp]
  \centering
  \caption{治療効果の比較(平均値 $\pm$ 標準偏差)}
  \label{tab:outcomes}
  \begin{tabular}{lccc}
    \hline
    & ベースライン & 3ヶ月後 & p値 \\
    \hline
    \multicolumn{4}{l}{\textbf{対照群}} \\
    代謝率(kcal/日) & 1650 $\pm$ 220 & 1620 $\pm$ 210 & 0.24 \\
    体重(kg) & 68.5 $\pm$ 12.3 & 68.8 $\pm$ 12.5 & 0.56 \\
    \hline
    \multicolumn{4}{l}{\textbf{治療群}} \\
    代謝率(kcal/日) & 1680 $\pm$ 230 & 1850 $\pm$ 240 & \textbf{0.001} \\
    体重(kg) & 69.2 $\pm$ 11.8 & 67.5 $\pm$ 11.2 & \textbf{0.03} \\
    \hline
  \end{tabular}
\end{table}

表\ref{tab:outcomes}に治療効果の詳細を示す。有意差のあるp値を太字で示している。

% 表3: 多変量解析の結果
\begin{table}[htbp]
  \centering
  \caption{SCDリスク因子の多変量解析}
  \label{tab:multivariate}
  \begin{tabular}{lccc}
    \hline
    リスク因子 & ハザード比 & 95\% CI & p値 \\
    \hline
    年齢(10歳増加) & 1.45 & 1.22--1.73 & \textbf{$<$0.001} \\
    男性 & 1.32 & 0.98--1.78 & 0.07 \\
    BMI(5 kg/m²増加) & 1.18 & 0.95--1.47 & 0.14 \\
    高血圧 & 1.67 & 1.23--2.27 & \textbf{0.001} \\
    糖尿病 & 1.89 & 1.34--2.66 & \textbf{$<$0.001} \\
    喫煙 & 1.52 & 1.15--2.01 & \textbf{0.003} \\
    \hline
  \end{tabular}
\end{table}

表\ref{tab:multivariate}に示すように、年齢、高血圧、糖尿病、喫煙が独立したSCDリスク因子であることが明らかになった。

\section{考察}
\subsection{主要な知見}
ここに日本語で主要な知見を記述します。

\subsection{先行研究との比較}
ここに日本語で先行研究との比較を記述します。

\subsection{本研究の限界}
ここに日本語で本研究の限界を記述します。

\section{結論}
ここに日本語で結論を簡潔に記述します。研究の主要な発見と、その臨床的・学術的意義をまとめます。

% 参考文献
\bibliographystyle{plain}  % 参考文献スタイル(plain, unsrt, alpha, abbrv など)
\bibliography{references}   % references.bibファイルを参照

\end{document}
