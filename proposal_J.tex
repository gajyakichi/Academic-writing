% ============================================================
% 研究計画書テンプレート(日本語版)
% ============================================================
% これは研究計画書のテンプレートファイルです
% 各セクションを自分の研究内容に置き換えてください
%
% コンパイル方法:
%   ./compile_J.sh(main_J.texと同じ方法)
% ============================================================

\documentclass[12pt, a4paper]{article}

% 日本語対応
\usepackage[japanese]{babel}

% 基本パッケージ
\usepackage{amsmath, amssymb}  % 数式
\usepackage{graphicx}          % 図の挿入
\usepackage{hyperref}          % ハイパーリンク
\usepackage{cite}              % 引用の最適化
\usepackage{longtable}         % 長い表
\usepackage{array}             % 表の拡張

% ページ設定
\usepackage[top=25mm, bottom=25mm, left=25mm, right=25mm]{geometry}

% ============================================================
% タイトル情報 - ご自身の情報に置き換えてください
% ============================================================
\title{研究計画書\\[0.5em]{\large 代謝とSCDに関する前向きコホート研究}}
\author{山口怜\\医学研究科}
\date{\today}

\begin{document}

\maketitle

\tableofcontents
\newpage

\section{研究の背景}

\subsection{問題の重要性}

心血管疾患は世界的な主要な死因であり、その中でも心臓突然死(Sudden Cardiac Death; SCD)は予後不良な病態である。近年、代謝機能とSCDリスクの関連が注目されているが、その詳細なメカニズムは十分に解明されていない。

\subsection{先行研究のレビュー}

これまでの研究により、以下のことが明らかになっている:

\begin{itemize}
  \item 代謝異常はSCDリスクの独立した予測因子である\cite{example_article}
  \item 基礎代謝率の低下は心血管イベントと関連する\cite{example_book}
  \item しかし、介入研究は限定的である
\end{itemize}

\subsection{知識のギャップ}

先行研究の限界として、以下の点が挙げられる:

\begin{enumerate}
  \item 縦断的なデータが不足している
  \item 介入効果を評価した研究が少ない
  \item 日本人を対象とした大規模研究がない
\end{enumerate}

本研究は、これらのギャップを埋めることを目的とする。

\section{研究の目的}

\subsection{主要目的}

代謝機能改善介入がSCDリスクに与える影響を、前向きコホート研究により明らかにする。

\subsection{副次的目的}

\begin{itemize}
  \item 代謝指標とSCDリスクの用量反応関係を評価する
  \item 介入効果の持続性を検討する
  \item サブグループ解析により効果修飾因子を同定する
\end{itemize}

\subsection{研究仮説}

\textbf{主仮説:}代謝機能改善介入により、対照群と比較してSCD発生率が30\%以上減少する。

\textbf{副次仮説:}介入効果は、年齢、性別、既往歴により修飾される。

\section{研究方法}

\subsection{研究デザイン}

\begin{itemize}
  \item \textbf{研究タイプ:}多施設共同前向きコホート研究
  \item \textbf{研究期間:}2026年4月〜2029年3月(3年間)
  \item \textbf{フォローアップ期間:}24ヶ月
\end{itemize}

\subsection{対象集団}

\subsubsection{選択基準}

\begin{itemize}
  \item 年齢:40〜75歳
  \item 代謝症候群の診断基準を満たす者
  \item 書面による同意が得られた者
\end{itemize}

\subsubsection{除外基準}

\begin{itemize}
  \item 既知の重篤な心疾患を有する者
  \item 悪性腫瘍の既往がある者
  \item 研究への参加が困難と判断される者
\end{itemize}

\subsection{サンプルサイズ設計}

必要サンプルサイズの計算:

\begin{itemize}
  \item 予想されるSCD発生率(対照群):10\%/2年
  \item 介入による相対リスク減少:30\%
  \item 検出力:80\%
  \item 有意水準:両側5\%
  \item 脱落率:15\%
\end{itemize}

計算の結果、\textbf{各群500名、計1,000名}が必要と算出された。

\subsection{介入内容}

\subsubsection{介入群}

\begin{enumerate}
  \item 個別化された運動プログラム(週3回、60分)
  \item 栄養指導(月1回、管理栄養士による)
  \item 行動変容支援(スマートフォンアプリ使用)
\end{enumerate}

\subsubsection{対照群}

\begin{itemize}
  \item 通常ケア
  \item 6ヶ月ごとの定期健診
  \item パンフレットによる情報提供
\end{itemize}

\subsection{評価項目}

\subsubsection{主要評価項目}

24ヶ月間のSCD発生率(心停止による突然死または蘇生された心停止)

\subsubsection{副次評価項目}

\begin{itemize}
  \item 総死亡率
  \item 心血管イベントの発生率
  \item 代謝指標の変化(体重、BMI、基礎代謝率)
  \item QOLスコアの変化
\end{itemize}

\subsection{データ収集}

\begin{table}[htbp]
  \centering
  \caption{データ収集スケジュール}
  \label{tab:schedule}
  \begin{tabular}{lccccccc}
    \hline
    評価項目 & ベースライン & 3ヶ月 & 6ヶ月 & 12ヶ月 & 18ヶ月 & 24ヶ月 \\
    \hline
    身体計測 & ○ & ○ & ○ & ○ & ○ & ○ \\
    血液検査 & ○ & & ○ & ○ & & ○ \\
    心電図 & ○ & & ○ & ○ & & ○ \\
    代謝測定 & ○ & & ○ & ○ & & ○ \\
    QOL調査 & ○ & & ○ & ○ & & ○ \\
    \hline
  \end{tabular}
\end{table}

\subsection{統計解析}

\subsubsection{主要解析}

\begin{itemize}
  \item Intention-to-treat解析を主解析とする
  \item Kaplan-Meier法によるイベント発生率の推定
  \item Log-rank検定による群間比較
  \item Cox比例ハザードモデルによる多変量解析
\end{itemize}

\subsubsection{副次解析}

\begin{itemize}
  \item Per-protocol解析
  \item サブグループ解析(性別、年齢層、併存疾患別)
  \item 感度分析
\end{itemize}

\subsubsection{統計ソフトウェア}

R version 4.3以降、有意水準は両側5\%とする。

\section{倫理的配慮}

\subsection{倫理審査}

本研究は、所属機関の倫理委員会の承認を得た後に開始する。ヘルシンキ宣言および「人を対象とする生命科学・医学系研究に関する倫理指針」を遵守する。

\subsection{インフォームドコンセント}

\begin{itemize}
  \item 研究参加前に、十分な説明を行う
  \item 書面による同意を取得する
  \item 研究参加はいつでも撤回可能であることを説明する
\end{itemize}

\subsection{個人情報保護}

\begin{itemize}
  \item データは匿名化して管理する
  \item 個人情報は施錠可能な場所に保管する
  \item データアクセスは研究責任者が管理する
\end{itemize}

\subsection{利益相反}

本研究に関連する利益相反はない。

\section{研究スケジュール}

\begin{table}[htbp]
  \centering
  \caption{研究スケジュール}
  \label{tab:timeline}
  \begin{tabular}{lp{10cm}}
    \hline
    時期 & 内容 \\
    \hline
    2026年4月〜6月 & 倫理審査、研究準備、スタッフトレーニング \\
    2026年7月〜12月 & 対象者リクルート、ベースライン評価 \\
    2027年1月〜2028年12月 & 介入実施、フォローアップ \\
    2029年1月〜3月 & データ解析、論文執筆 \\
    \hline
  \end{tabular}
\end{table}

\section{予算計画}

\subsection{必要経費}

\begin{table}[htbp]
  \centering
  \caption{予算内訳}
  \label{tab:budget}
  \begin{tabular}{lrr}
    \hline
    項目 & 単価(円) & 合計(円) \\
    \hline
    人件費 & & 10,000,000 \\
    \quad 研究補助員(2名、3年) & & 6,000,000 \\
    \quad データマネージャー(1名、2年) & & 4,000,000 \\
    \hline
    検査費用 & & 15,000,000 \\
    \quad 代謝測定(1,000名×3回) & 5,000 & 15,000,000 \\
    \hline
    介入費用 & & 8,000,000 \\
    \quad 運動指導 & & 5,000,000 \\
    \quad 栄養指導 & & 3,000,000 \\
    \hline
    その他 & & 2,000,000 \\
    \quad アプリ開発・運用 & & 1,500,000 \\
    \quad 消耗品費 & & 500,000 \\
    \hline
    \textbf{合計} & & \textbf{35,000,000} \\
    \hline
  \end{tabular}
\end{table}

\subsection{資金源}

\begin{itemize}
  \item 科学研究費補助金(予定)
  \item 学内研究費
  \item 民間財団研究助成(申請予定)
\end{itemize}

\section{期待される成果}

\subsection{学術的意義}

\begin{itemize}
  \item 代謝とSCDの因果関係の解明
  \item 介入効果のエビデンス創出
  \item 国際的なガイドライン策定への貢献
\end{itemize}

\subsection{社会的意義}

\begin{itemize}
  \item SCD予防戦略の確立
  \item 医療費削減への貢献
  \item 国民の健康寿命延伸
\end{itemize}

\subsection{成果発表計画}

\begin{itemize}
  \item 国際学会での発表(年1回以上)
  \item 査読付き論文の発表(計3報以上)
  \item 一般向けセミナーの開催
\end{itemize}

\section{研究の限界と対策}

\subsection{想定される限界}

\begin{enumerate}
  \item フォローアップ期間の脱落
  \item 介入の標準化の困難さ
  \item 外部妥当性の限界
\end{enumerate}

\subsection{対策}

\begin{itemize}
  \item 定期的な連絡による脱落防止
  \item 詳細な介入マニュアルの作成
  \item 多施設共同研究による一般化可能性の向上
\end{itemize}

\section{参考文献}

\bibliographystyle{plain}
\bibliography{references}

% 参考文献がない場合は、以下のように手動で記載することもできます
% \begin{thebibliography}{99}
% \bibitem{example_article}
% 著者名. 論文タイトル. ジャーナル名. 年;巻:ページ.
% \end{thebibliography}

\end{document}
